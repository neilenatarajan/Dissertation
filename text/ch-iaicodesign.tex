% \begin{savequote}[8cm]
% Alles Gescheite ist schon gedacht worden.\\
% Man muss nur versuchen, es noch einmal zu denken.

% All intelligent thoughts have already been thought;\\
% what is necessary is only to try to think them again.
%   \qauthor{--- Johann Wolfgang von Goethe \cite{von_goethe_wilhelm_1829}}
% \end{savequote}

\chapter{\label{ch:iaicodesign}A Human-Centric Approach to Considering Diversity in Talent Identification [WIP]} % Maybe this is two chapters? The co-design could be one chapter, and the SPF itself could be another?

% mention that these are Intrinsically Interpretable Models, as this ties it to the previous two chapters


\minitoc

\section{Introduction}
(To-do)

\section{Background}
(To-do)

\subsection{Diversity as an Institutional Value}
(This section will cover where DEI comes from, how it's been institutionalised, critiques, legal complications, and how I am approaching it.)
(To-do)

% conservative critiques as `too diverse' david goodhart

% combahee river collective `identity politics' `The term “identity politics” was first popularized by the 1977 manifesto of the Combahee River Collective, an organization of black feminist activists.'
% intersectionality (pre-crenshaw?)
% standpoint epistemology
% zetetic 
% `Reasons to think the oppressed have such an advantage include that they tend to have: more informative experiences of oppression (e.g. Mills 1998), greater incentives to learn about and criticize oppression (e.g. Jaggar 1983), and access to networks of other oppressed people who are themselves epistemically advantaged with respect to oppression (e.g. Dror 2023).' tuckwell and yeo

% wheel of academic privilege example

% neoliberal multiculturalism produces racialised distinctions of `good diversity' (a racially diverse football team) and `bad diversity' (racialised subjects criticising foreign policy are seen as `failing to integrate') - lentin and titley 2011

% ` diversity is a malleable discourse suffused through
% a variety of institutional practices and political frameworks. Diversity may be a
% hybrid product of strands of contemporary thinking on identity, difference, power
% and social justice, but this does not entail that discourses and practices of diversity
% offer the enabling or subversive possibilities associated with it in all or even many
% contexts.' lentin and titley - benneton

% the diversity bargain - warikoo

% \subsection{Diversity Concepts and Measures}
% draw significantly from Steel et al

% in science

% budescu
% steel et al
% schumann
% entrofy - huppenkothen et al

% \subsection{Employee diversity for productivity}
% page et al the diversity bonus, the difference
% ostergaard

% \subsection{Fair ML}
% Group fairness
% Individual fairness - binns, fleisher

% \subsection{Critiques of Diversity}
% Taiwo and Rossi
% Ahmed:
% on being included: `commitment to diversity is frequently substituted for a commitment to actual change'
% documenting rather than doing

% diversity training - not clear if it works `We suggest that the enthusiasm for, and monetary investment in, diversity training has outpaced the available evidence that such programs are effective in achieving their goals.' devine et al

\section{Methods Part 1}
\subsection{Researcher Bias}
Following the methodology of \textcite{braun_using_2006}, we acknowledge that our own biases may (positively or negatively) influence our research. We are a team of three researchers: one South Asian man from the US, one South Asian woman from India, and one white man from the UK.

\subsection{Our Study}
Our study is broken into two components: a series of interviews with talent identification professionals and two participatory design workshops with said professionals. In the first session, we seek to ascertain how these professionals understand diversity, how they operationalise it in their work, and how they envision using technology to assist them in that process. In the second session, we show these professionals a series of mock-up designs drawn in response to the interviews, then we aim to collaboratively design tools that can help them better operationalise diversity.

\subsection{Interviews and Thematic Analysis}
Our interviews aim to answer three research questions:

\begin{enumerate}
    \item What is diversity?
    \item Which elements of diversity matter in a talent identification context? Why?
    \item How would you envision technology assisting you in operationalising diversity?
\end{enumerate}

In answering our first set of research questions, we follow \cite{braun_using_2006}'s methodology for thematic analysis. We first conduct 45-minute semi-structured interviews with 15 talent identification professionals. In each interview, we first ask general questions about their selection methodology; we next ask specifically about diversity and it's role in selection; we move on to a ``crazy 8s'' exercise designed for participatory design; we conclude with a ``magic app'' exercise, also designed for participatory design. A question-by-question protocol for these interviews is found in Appendix \ref{app:protocolmockup}. However, following the methodology of \cite{braun_using_2006}, we do not limit our analysis to these questions. Instead, we deviate from this script as guided by the conversations and our overarching research questions, then we allow themes to emerge naturally from the data.

After interviewing participants, the interviewer transcribes the interviews and anonymizes them. Two researchers then independently ``open-code'' each anonymized transcript, looking for anything relevant to our research questions. The researchers meet to discuss their open codes, and collaborate in the process of clustering these codes into subthemes and themes.

\subsection{Design Workshops}
Before this workshop, we use the findings from the thematic analysis to design a series of mock-up technologies. Our research question for this workshop is, for each mock-up: ``How can this mock-up help you better promote diversity in talent identification?''. A full protocol for this workshop is found in Appendix \ref{app:protocolmockup}.

After running this workshop, the researchers use written notes and meeting recordings to draw conclusions about the relative utility of each mock-up, and extract guidelines for the design of future tools.

\section{Results Part 1}
% \begin{center}
%     \label{tab:allthemes}
%     \begin{tabular}{| l | l | l | l | l | l |} 
%         \textbf{Why Diversity} & \textbf{Types of Diversity} & \textbf{Operational Risks and Considerations} & \textbf{Fairness and Bias} & \textbf{Scholarship Goals} & \textbf{Merit}\\
%         Different perspectives & Socioeconomic & Outreach & Fairness & Impact & Performance relative to disadvantage\\
%         \emph{...In the same room} & \emph{Parental income} & Support & \emph{...to all applicants} & \emph{...on all applicants} & Measurement\\
%         Representativeness & \emph{Parental education} & \emph{...during the application process} & \emph{...to the selected scholars} & \emph{...on the selected scholars' performance} & Merit vs. diversity\\
%         \emph{...of a general population} & \emph{Generational wealth} & \emph{...after selection} & \emph{...to the scholarship} & \emph{...on the selected scholars' opportunities} & \\
%         \emph{...of the eligible population} & Sex, gender, and sexuality & Selectors & Bias & \emph{...by the selected scholars on the world} & \\
%         \emph{...of the applicant population} & \emph{Sex} & Applicant fraud & \emph{Measurement bias} &  & \\
%         \emph{...of a taget population } & \emph{Gender identity} &  & \emph{Decision-makers' bias (prejudicial)} &  & \\
%         Boosting disadvantaged groups & \emph{Sexual orientation} &  & \emph{Decision-makers' unique perspective (probative)} &  & \\
%          & Geography &  & Transparency &  & \\
%          & \emph{Nationality} &  & \emph{Transparency as fairness} &  & \\
%          & \emph{Global south} &  & \emph{...during the application process} &  & \\
%          & \emph{Region} &  & \emph{...after selection} &  & \\
%          & Types of thinking &  &  &  & \\
%          & \emph{Subject area interest} &  &  &  & \\
%          & \emph{Personality type} &  &  &  & \\
%          & \emph{Core beliefs} &  &  &  & \\
%          & \emph{Problem solving approaches} &  &  &  & \\
%          & \emph{Political views} &  &  &  & \\
%          & Race &  &  &  & \\
%          & \emph{International categorisations of race} &  &  &  & \\
%     \end{tabular}
% \end{center}


\subsection{Participants}
We interview talent identification professionals (N=15) from two talent investment programs. These professionals provided information on their roles in the selection process, and optionally provided demographic information; this information is shown in Tables \ref{tab:roles} and \ref{tab:demo}.

\begin{center}
    \label{tab:roles}
\end{center}

\begin{center}
    \label{tab:demo}
\end{center}


\subsection{Why Diversity}
A central theme of our investigation was the question of why diversity matters. To this end, we asked questions like ``What is diversity?'' and ``Why does diversity matter?''. We found that answers to these questions were closely related, and were able to cluster answers to these questions into three central subthemes: `different perspectives', `representativeness', and `boosting disadvantaged groups'. These are listed as subthemes of `Why Diversity' in Table \ref{tab:whydiv}.

\begin{table}[h]
    \centering
    \caption{Themes and Subthemes Related to What Diversity is and Why Diversity Matters}
    \label{tab:whydiv}
    \begin{tabular}{|l|} 
        \hline
        \textbf{Why Diversity} \\
        \hline
        Different perspectives \\
        \emph{...In the same room} \\
        Representativeness \\
        \emph{...of a general population} \\
        \emph{...of the eligible population} \\
        \emph{...of the applicant population} \\
        \emph{...of a taget population } \\
        Boosting disadvantaged groups \\
        \hline
    \end{tabular}
\end{table}

\subsubsection{Different Perspectives}
The first subtheme we identified was `different perspectives'. Participants frequently mentioned that diversity was important because it brought different perspectives into the same room. This was seen as important for a few related reasons, e.g., the ability to see problems from different angles and the ability to make better decisions. Several participants referred to the: ``benefits of diverse perspectives''. One wrote, when discussing their personal experience working with winners in a talent investment program, that there is: ``magic happening with lots of...diverse perspectives in the room''. Peoples' experiences were particularly relevant here. As one participant writes: ``you want to have diverse perspectives from people who look different with different experiences''.

\subsubsection{Representativeness}
The second subtheme we identified was `representativeness'. This, we observed, was often spoken of in relationship to a larger population. Most frequently, participants spoke of the importance of having a cohort that was representative of the `eligible population'. I.e., one participant said: ``[You want] a community which is representative of where you are selecting young people from''. Others spoke of this in broader, more general terms: ``you have that...broad...representation of people''. Participants identified the importance of building a cohort that variously: ``reflects the diversity of the countries'' and is ``more representative of the national population than the STEM field already is'' (the latter participant was speaking specifically of selecting STEM applicants).Others talk about representation of a particular target population, i.e.: ``representation...because that gives you insight for the people that you're trying to serve,'' and ``you have to be....reflective of your market''. Finally, selectors discussed the importance of representing an applicant population: ``[We want] a cohort that is representative of the pool''.

\subsubsection{Boosting Disadvantaged Groups}
The final subtheme we identified was `boosting disadvantaged groups'. This was often spoken of in individual terms, speaking of particular applicants in need of support. I.e.: ``identify those talents and specifically boost up people who are in need of support''. One participant identified `boosting' as a key metric for the programme: ``we need to know that...we have some level of...impact here, and...if all we're doing is supporting someone who is already on an amazing trajectory and then maybe that means we're not altering their trajectory at all. That's a question of efficiency of our dollars.''

However, this individualistic metric for success was often also identified with underrepresented or disadvantaged groups. One participant said: ``The focus on gender has been to give the sex that has had the least opportunity the opportunity in this program.''

\subsubsection{Relationships Between Subthemes}
Many selectors identified the word `diversity' with the `representativeness' subtheme. For example, one said: ``if you're doing scholarship programmes, the world is diverse. So if you want to do a global programme, then you have to be diverse''. Nonetheless, these selectors also variously identified our other subthemes as reasons diversity is important. The same selector said: ``The purpose of diversity on the gender aspect was to make sure that biological male female were getting equal opportunities''.

\subsection{Types of Diversity}
Another central focus of our investigation was on different types of diversity. We asked participants to break down their understanding of diversity into different elements, and to discuss why these elements were important. We found that participants identified a wide range of different types of diversity, which we clustered into a series of subthemes. These subthemes are listed as subthemes of `Types of Diversity' in Table \ref{tab:typesdiv}.

\begin{table}[h]
    \centering
    \caption{Themes and Subthemes Related to Types of Diversity}
    \label{tab:typesdiv}
    \begin{tabular}{|l|}
        \hline
        \textbf{Types of Diversity} \\
        \hline
        Socioeconomic \\
        \emph{Parental income} \\
        \emph{Parental education} \\
        \emph{Generational wealth} \\
        Sex, gender, and sexuality \\
        \emph{Sex} \\
        \emph{Gender identity} \\
        \emph{Sexual orientation} \\
        Geography \\
        \emph{Nationality} \\
        \emph{Global south} \\
        \emph{Region} \\
        Race \\
        \emph{International categorisations of race} \\
        Types of thinking \\
        \emph{Subject area interest} \\
        \emph{Personality type} \\
        \emph{Core beliefs} \\
        \emph{Problem solving approaches} \\
        \emph{Political views} \\
        \hline
    \end{tabular}
\end{table}


Notably, in addition to the standard demographic categories commonly considered `demographic diversity', participants identified a wide range of other types of diversity commonly termed `cognitive diversity' \cite{page2019diversity}. These included `subject areas of interest', `personality type', `core beliefs', `problem solving approaches', and `political views'.

\subsubsection{Socioeconomic}
Participants frequently identified socioeconomic diversity as particularly important in the context of a talent investment program. Indeed, one participants said: ``Socioeconomic [diversity] is probably the most important''. Another say: ``Socioeconomic background is like number one from my perspective''. This was identified as particularly important for several reasons. Participants stated: ``Because right now the SAT, for example, is more highly correlated to socioeconomic status than it is to anything else'', and: ``It's a scholarship scheme, so I think it should be for kids who cannot afford normally the fees at the university''.

Outside of the standard categorisations by income and wealth, participants also identified ``familial education level'' or ``family education backgrounds'' as a particularly important metric for understanding socioeconomic diversity in a scholarship context. One participant noted that socioeconomic status varied in both meaning and measurement from country to country: ``For example, in Columbia, there's a whole society to organise on a 1 to 7 scale for socioeconomic status''.

\subsubsection{Sex, gender, and sexuality}
While all participants noted some manner of sex, gender, and sexuality diversity as important, participants disagreed on the relative emphasis that should be place on each. One participant noted: ``[Sex] is important. I think it will get diluted if we focus on identity gender because....the purpose of diversity on the gender aspect was to make sure that [men and women] were getting equal opportunities''. Another noted that, while sexual identification diversity was important in other contexts, they ``wouldn't select for that'' in this context. Others listed ``Sexual orientation'' and ``Gender'' as important metrics for understanding diversity in a scholarship context. However, save for the participant who noted the distinction between sex and identity gender, participants expressed reluctance to discuss the relationships between this difficult concepts.

\subsubsection{Geography}
Participants frequently mentioned the importance of ``[A] wide array of different geographical [representation]''. Others mentioned a ``Regional distribution'' alongside the geographic one.

In particular, emphasis was placed on geographic markers of socioeconomic status such as  the ``Global South'', ``Indigenous communities'', and ``Low income countr[ies]''. One participant noted: ``Immigration status is tied so closely to socioeconomic status'', while another noted that ``[Geography] is connected to socioeconomics because we know there are some poorer country and rich countries''. Others still asked questions like ``Do they have a passport?'' and ``Are they in a refugee camp?''.

Furthermore, participants saw it as important that their programmes had ``Global reach''. They expressed a desire for: ``Diversity of people coming from variety of places''.

\subsubsection{Race}
While many participants identified race as an important dimension of diversity, none suggested we explicitly select for racial diversity. Several participants instead noted the difficulty of measuring race in a global context: ``Racial categories obviously vary by country''. One participant noted: ``[In places] like Brazil or England, there's a different categories of race than there are in the US...[In Brazil] there's a board of people who decide what people's race are''. In a global context, however, many participants pointed to relationships between geography and race, and hence suggested diversifying across geography in place or race: ``If it's an international programme then you can use geography as a proxy''.

\subsubsection{Types of Thinking}
One participant noted that: ``You want as much representation from different different types of thinking as you can, because I want perspectives to be listened to equally''. This manifested in many ways.

Participants tended to express the belief that personality-type-diversity could improve group cohesion: ``With that is understanding personality types be able to tell which two like which people would get on well with each other.'' One participant suggested a ``Personality test'', and another specifically mentioned a desire to diversify across ``Openness''.

However, while personality type was seen as important, core beliefs were seen as even more so. One participant noted and interest in diversity of ``Interests politically''. Another expressed a desire for diversity of ``peoples core beliefs are that's separate from religion'', and also noted: ``I would try to have a good representation of....religious groups''.

\subsection{Operational Risks and Considerations}
While our study did not focus on the operational aspects of selection, several selectors' understanding of diversity was closely tied to the operational realities of selecting for and running a scholarship. In answering our questions, several participants identified operational risks or considerations that impacted their understanding of diversity. These are listed as subthemes of `Operational Risks and Considerations' in Table \ref{tab:operational}.

\begin{table}[h]
    \centering
    \caption{Themes and Subthemes Related to Operational Risks and Considerations}
    \label{tab:operational}
    \begin{tabular}{|l|}
        \hline
        \textbf{Operational Risks and Considerations} \\
        \hline
        Outreach \\
        Support \\
        \emph{...during the application process} \\
        \emph{...after selection} \\
        Selectors \\
        Applicant fraud \\
        \hline
    \end{tabular}
\end{table}

\subsubsection{Outreach}
While our study was focused primarily on selecting a diverse cohort from a fixed applicant pool, several participants immediately began answering questions from the perspective of outreach with the goal of growing a more diverse pool of applicants to select from. In particular, participants suggested that ``Using technology for....targeted outreach'' could help improve overall cohort diversity before selection even begins. One said: ``Giving you very clear signposting on where you may want to focus, you know further recruitment or outreach or whatever it might be to make sure that your programme is diverse at the end of the day''. Another added: ``You can target your outreach dollars to communities where you know that underrepresented talent exists''.

\subsubsection{Support}
Similarly, participants suggested that technology could enable the support of applicants from underrepresented groups, which would also improve diversity. One participant suggested that technology could be used to provide: ``[Support] to keep people that you're attracting from underrepresented backgrounds and help them get across the the finish line''. While others focused on the: ``Support [applicants] need to actually get through your programme'', i.e., supporting applicants after acceptance. Another suggested that technology could be used to provide support to applicants ``after selection''.

\subsubsection{Selectors}
Other applicants noted that diversity did not apply merely to applicants. Instead, for programs where large groups of selectors assist in the selection process, ``Tracking the diversity of of the selectors'' and ``[Monitoring] how their scoring and reviewing applicants [for] prejudice or biases''.

\subsubsection{Applicant fraud}
Finally, a persistent concern with selecting based on particular diversity characteristics, especially self-reported metrics of diversity characteristics, was the potential for applicant fraud. One participant requested: ``A fraud detector'', while another expressed a desire to ensure that the process ``Isn't super gameable''.

\subsection{Fairness and Bias}
Though not a type of diversity as we have understood it here, many applicants referenced similarities between diversity metrics and metrics of fairness or bias. Furthermore, several suggested that improving fairness while reducing bias would likely yield a more diverse cohort. These themes are reflected in Table \ref{tab:fairnessbias}.

\begin{table}[h]
    \centering
    \caption{Themes and Subthemes Related to Fairness and Bias}
    \label{tab:fairnessbias}
    \begin{tabular}{|l|}
        \hline
        \textbf{Fairness and Bias} \\
        \hline
        Fairness \\
        \emph{...to the applicants} \\
        \emph{...to the world} \\
        Bias \\
        \emph{Measurement bias} \\
        \emph{Decision-makers' bias (prejudicial)} \\
        \emph{Decision-makers' unique perspective (probative)} \\
        \hline
    \end{tabular}
\end{table}

\subsubsection{Fairness}
Participants discussed that it was important that applicants: ``Get fair chance on their on their academic merit''. This translated to an emphasis on ``Fairness in the assessment''.

However, participants also noted that ``The way the world works is unfair'', and found it important that the programme is: ``Making sure that the world is fairer by bringing more diversity to this world''. In this way, participants found: ``[The] representative thing....goes back to fairness''. One participant noted that: ``Affirmative action....can come across as unfair to some people, but....it's trying to balance things out when things have been so unequal for so long''.

\subsubsection{Bias}
Participants also discussed bias as both a human- and machin-decision-making problem. Many participants appealed to technology's ability to be comparatively impartial as an important mitigator of bias, I.e., one participant repeatedly requested a: ``Non-biased program''; another stated a preference for: ``'ata analysis to make decisions on who we should be supporting as opposed to having humans try to make those decisions with all their biases''. However, others noted that common machine decision-making paradigms amplify bias, and that it was important to be aware of this: ``AI has a lot of bias in it''.

Others noted the possibility for technology to elucidate biases in both humans and machines. One participant requested: ``Some kind of tool that can detect bias in a selection''.

\subsection{Programme Goals}
Talent identification professionals from both programmes identified a series of goals for their programmes. These goals were discussed by both groups as an intended form of ``impact'' and both groups closely related achieving their goals to their answers to why diversity mattered. As these programmes wish to remain anonymous, we refrain from detailing their goals themselves, but we include this theme for reference in Table \ref{tab:scholarshipgoals}.

\begin{table}[h]
    \centering
    \caption{Themes and Subthemes Related to Programme Goals}
    \label{tab:scholarshipgoals}
    \begin{tabular}{|l|}
        \hline
        \textbf{Scholarship Goals} \\
        \hline
        Impact \\
        \emph{...on all applicants} \\
        \emph{...on the selected scholars' performance} \\
        \emph{...on the selected scholars' opportunities} \\
        \emph{...by the selected scholars on the world} \\
        \hline
    \end{tabular}
\end{table}

\subsubsection{Impact}
Participants found key goals of their program to include: ``Orient[ing] [scholars] towards social impact or using their talent for good''. They tended to encourage: ``[Scholars'] working towards something impactful over the course of their career''.

\subsection{Merit}
Finally, many participants reflected on the relationship between merit and diversity. While some participants saw these as competing goals, others saw them as complementary. In particular, complementary views often viewed merit as a form of performance relative to specific advantages, or noted that many of our measurement tools are biased across our chosen diversity dimensions. These themes are reflected in Table \ref{tab:merit}.

\begin{table}[h]
    \centering
    \caption{Themes and Subthemes Related to Merit}
    \label{tab:merit}
    \begin{tabular}{|l|}
        \hline
        \textbf{Merit} \\
        \hline
        Performance relative to disadvantage \\
        Measurement \\
        Performance vs. diversity \\
        \hline
    \end{tabular}
\end{table}

\subsubsection{Performance relative to disadvantage}
Many participants identified merit as something difficult to disentangle from performance. One participant noted that applicants may appear less qualified because they: ``Didn't have the chance; didn't have the opportunities'', while others with the opportunities will appear more qualified. Another participant began by asking: ``How good is their 3A stars based on where they've come from?'', then proceeded to reflect that ``Your performance relative to your opportunity or maybe expected performance'' is a key indicator of merit.

\subsubsection{Measurement}
Closely related, participants questioned our ability to measure merit independent of opportunity: ``[Whether they perform well] because they have the opportunity or because they are brilliant – I think that these two are really difficult to untangle.'' Another noted that ``Contextual factors mess up our otherwise seemingly objective measures of merit....national context and family income is messing up your ability to measure the thing you actually care about''. They continued to note that they: ``Need to pay attention to [diversity] because it's messing up your measures of what you actually care about''.

\subsubsection{Performance vs. diversity}
Finally, participants noted occasions where performance and diversity were ostensibly competing goals. However, even here, participants recognised that observed performance and actual merit may differ. One participant noted that: ``Overriding aim is for [the programme] to be as diverse as is possible but still meet a standard....relative score of like how good their application is based on all these kind of contextual factors''. Another requested a technology that helps discover how: ``Close you are to your idealised diversity targets and how close you are to maximising whatever it is you think your your you're maximising in your in your performance scores''.

\section{Methods (Part 2)}
\subsection{Design Mock-ups}
We apply the results of our thematic analysis to design a series of mock-up technologies. These technologies help talent identification professionals better understand and operationalise diversity in their selection processes. We then present these mock-ups to participants in participatory design workshops. As the participants all come from two separate talent investment programs, we present run one workshop for each group. Furthermore, before presenting to the broad audience of each group, we run two `pilot' workshops, where we tweak the mock-ups based on feedback from a small group of participants. Thus, while the prototypes are broadly similar between the two groups, minor differences occur between prototypes shown to each group. These prototypes are shown in Appendix \ref{app:protocolmockup}.

In particular, while we sought to draw a distinction between ``representativeness'' and ``entropy'' measurements of diversity in Figures \ref{fig:representativeness} and \ref{fig:entropy} (based on the ``Representativeness'' and the ``Different perspectives'' themes, respectively), one talent investment organisation requested that we elide that distinction so as to better fit with their current selection process. Thus, though we have three cohort-level-mockups, we present Figures \ref{fig:representativeness} and \ref{fig:entropy} to one group, and we present Figure \ref{fig:diversity} to the other. We present all three of Figures \ref{fig:demographic}, \ref{fig:impact}, and \ref{fig:advantage} to both groups. (Note here that Figures \ref{fig:demographic} and \ref{fig:impact} are based on the ``Representativeness'' theme, while Figure \ref{fig:advantage} is based on the ``Boosting disadvantaged groups'' theme.)

\subsection{Design Workshops}
Our central research question for this workshop is, for each mock-up: ``How can this mock-up help you better promote diversity in talent identification?''. In each workshop, we ask participants to consider each mock-up in turn, and to discuss how they might use it in their selection process. We then ask participants to consider how these mock-ups might fit into their current selection process, and how they might change their process to better incorporate these mock-ups. Finally, we ask participants to consider how their current selection process might make best use of these mock-ups, and whether they think these mock-ups would be beneficial. A question-by-question protocol for these workshops is found in Appendix \ref{app:protocolmockup}.

After running this workshop, the researchers use written notes and meeting recordings to draw conclusions about the relative utility of each mock-up, and extract guidelines for the design of future tools.

\section{Results Part 2}
\subsection{Cohort-Level Tools}
We can separate the mock-ups in Figure \ref{fig:mockups} into `cohort-level tools' (Figures \ref{fig:representativeness}, \ref{fig:entropy}, and \ref{fig:diversity}) and `individual-level tools' (Figures \ref{fig:demographic}, \ref{fig:impact}, and \ref{fig:advantage}). We consider cohort-level tools first.

\subsubsection{Representativeness}
The first cohort-level tool we presented to participants was a tool designed to help them understand the representativeness of their cohort, i.e., Figure \ref{fig:representativeness}. This tool was designed to help participants understand how well different possible cohorts achieved certain program-defined representativeness standards. Example standards included a preference for gender parity, inclusion of applicants from as many nations as possible, a preference for applicants from first-generation households, and a preference for applicants of below-average income. It should be noted that, as we set targets for representation of low-socioeconomic-status applicants, Figure \ref{fig:representativeness} also helps applicants consider our ``Boosting disadvantaged groups'' theme.

They noted a tradeoff in the tool between representativeness of a cohort and a cohort's average scores, at least at the frontier. However, participants' experience in selection supported the existence of such a trade-off: ``Real decision-making always sees tradeoffs like these'', one said; another remarked: ``[It] makes sense that top scoring candidates don't necessarily help you build the most diverse cohort''. However, participants were confused by the relative framing, wherein the highest scoring cohort had a ``0\%'' representativeness score (and vice-versa). ``Why does the top scoring cohort have 0\% representativeness?'' was a common question.

On the whole, however, participants found this tool useful. One participant noted that: ``This chart helps you figure out the level of compromise you're willing to make on both axes''. Another noted: ``If this chart were real (rather than hypothetical), and you could see who you were losing, this would be useful''. In the hypothetical, however, participants settled closer to the centre of the frontier: ``[Our] target here is to look somewhere [from] red to yellow''.

Furthermore, participants expressed desires to replace example axes with axes that better reflected their own selection criteria. On the representativeness axis, one participant stated: ``Let's track but not use first-gen[eration university students]''. while another noted: ``For people working together, it's useful to have someone who is that `glue'''. On the average overall score axis, participants questioned the relationships between subscores in the overall score: ``What is the relationship between the score variables?''. They also noted: ``One axis for [applicant scores] may not be enough''.

\subsubsection{Entropy}
The second cohort-level tool we presented to participants was a tool designed to help them understand the ``entropy'' of their cohort. This can be seen in Figure \ref{fig:entropy}. While ``entropy'' does appear in talent identification literature, it does not have a well-defined meaning in this context \cite{huppenkothen2020entrofy}. We define it here to formalise the notion of ``different perspectives''; it is the average number of relevant differences between pairs of members of a cohort. That is for any cohort $C$ and series of relevant traits $k$, we define entropy as:

\begin{equation}
\begin{split}
    \text{Entropy} &:= \text{mean}(d(x_i, x_j)| i, j \in C) \\
    d(a, b) &:= \sum_{k} \mathbb{I}(a_k \neq b_k)
\end{split}
\end{equation}

It should first be noted that this definition was unfamiliar to participants before this workshop – participants asked: ``Entropy is chaos in chemistry. How does this relate to our usage here?''.

As this mock-up and Figure \ref{fig:representativeness} were shown to the same group, the group was given the opportunity to compare the two prototypes. Participants asked about the relationship between scores on the two charts. Participants determined that, while Figure \ref{fig:representativeness} displayed ``diversity scores'', Figure \ref{fig:entropy} displays ``characteristics''. Participants took these characteristics to include certain personality traits, such as the ability to work well in a group (`glue'), and other program-specific desiderata. Furthermore, participants were interested to know the relationship between these mock-ups in practice: ``If we maximise based on [Figure \ref{fig:representativeness}] scores, what would the Entropy scores be?''.

While participants accepted easily that representativeness would exist on a frontier with average score, they lacked the intuition that entropy would have a similar relationship: ``[The frontier] makes sense when you're talking about [socioeconomic status], but makes less sense in other contexts''.

Finally, participants expressed an anxiety about measuring the relevant dimensions for entropy. ``What are our metrics and are they reliable?'' was echoed by several participants. One participant noted: ``If it's all self-report, then we can't do anything with it''. However, as this organisation's selection process includes an interview, participants noted: ``This is better post-interview than it is pre-interview'', as interviews will collect observational data on many of these characteristics. 

\subsubsection{Diversity}
While one group of participants was shown Figures \ref{fig:representativeness} and \ref{fig:entropy}, the other was only shown Figure \ref{fig:diversity}. It should be noted that, while we have replaced ``entropy'' or ``representativeness'' with ``diversity'' here, little else differs between Figures \ref{fig:representativeness} and \ref{fig:diversity}. 

However, the linguistic distinction between ``representativeness'' and ``diversity'' sparked a dramatic shift in the focus of discussion. While the discussion around Figure \ref{fig:representativeness} squarely centred the ``representativeness'' subtheme of diversity, discussion here centred the ``Boosting disadvantaged groups'' subtheme instead. Participants spent time debating ``Is the program needs-based or merit-based?''. They noted  that ``This chart helps'' facilitate that discussion.

More similar to the discussion of Figure \ref{fig:representativeness}, participants noted the `inverse relationship' between the diversity and average score axes. Again, though, participants' past selection experience supported the existence of this relationship: ``Some candidates get a big diversity boost and score terribly''. However, participants here disagreed on whether this was evidentiary of a real difference in performance or a bias in the selection process: ``No matter how we try to test in a way that's unbiased, there will still be bias. This implies we should pick far more towards the diverse end'', one participant argued; another participant argued: ``we should limit total possible diversity bonus'', in order to keep candidates with low scores from being selected.

Furthermore, participants noted that, in past selection processes, they had made tradeoffs between these two axes: ``5-10\% [of the cohort] really get to a struggle between quality and diversity''.

However, participants also note that, as some informal criteria are not scored here, this tool is limited in its usefulness. ``There are other factors that we don't formalise here. E.g., our targets....don't mention specific countries''. Most crucially, there are `qualitative' judgements made of candidates that cannot be captured here: ``We're using quantitative to sift through qualitative''.

\subsection{Individual-Level Tools}
\subsubsection{Applicant Demographic Information}
The first and simplest individual-level tool we presented to participants was a tool designed to help them visualise the demographic information of individual applicants alongside their scores. This can be seen in Figure \ref{fig:demographic}, where applicant demographics are included as tags, bolded based on their importance to the program, and colour-coded based on their prevalence in the current cohort. 

It should further be noted that Figure \ref{fig:impact} contains similar information to Figure \ref{fig:demographic}, though the former also contains explanations. Thus, participants were shown both together. We analyse them together here as well.

Participants often compared to two mockups. In particular, they preferred the more detailed Figure \ref{fig:impact} in discussions, but found it overwhelming in isolation. ``[We] can't send [Figure \ref{fig:impact}] as a pre-read, but [Figure \ref{fig:demographic}] makes more sense in isolation, so better for a pre-read''. On Figure \ref{fig:demographic} in particular, one participant noted: ``Helpful for the process, not so much for the [final cohort selection]''. Another said of Figure \ref{fig:impact}: ``This has the most potential at the later stages of decision-making''.

Participants simultaneously expressed gratitude that the mock-ups were as simplified as they were, and a desire for more detail. One participant noted that the mock-ups are: ``Very constrained in terms of what is being shown''. They later clarified that this was meant in a positive light: ``This doesn't include all of the factors, but for the decision that's good, because if prevents info overload''. However, participants also noted that the mock-ups were missing key information: ``We need to know more about the applicants' backgrounds''.

Several specific pieces of information were requested in this prototype. These include: ``comments from selectors'', ``a narrative summary...written by the selector team'', ``advantage score'', ``impact on entropy''. some program-specific desiderata, and some demographic characteristics otherwise left out. Finally, participants requested that we, ``Summarise these scores together'', to get, ``A composite impact on cohort diversity''.

\subsection{Applicant Advantage Scores}
The final individual-level mock-up shown to participants was a tool designed to help them understand the relative advantage or disadvantage of individual applicants and contextualise their scores relative to advantage; this was designed to help participants with `Boosting disadvantaged applicants'. This mock-up can be seen in Figure \ref{fig:advantage}.

In short, we formalise a metric for `advantage' based upon socioeconomic factors, then residualise applicants' scores so they can be compared fairly across advantage levels. The residualisation process here involves constructing a linear model of scores based on advantage, then subtracting the predicted score from the actual score. This is a common technique in the social sciences.

Though some participants had dealt with residualised scores in the past, they still expressed confusion at the way this information was displayed. ``This is probably confusing'', one participant stated, ``[It] presents more information when there's already a lot''. Another noted: ``The adjustments are overengineered'' (several participants agreed with this.)

Other participants noted that the adjustments alone do not guarantee the boosting of underprivileged applicants. Concerns emerged over both over- and under-adjustment: ``There's a risk of over-adjusting, i.e., social engineering''; ``It's possible that `adjustment' doesn't do anything if low and high SES people are doing equally well''. Furthermore, one participant noted that: ``You could end up with a bunch of high-SES people post-adjustment, so this doesn't actually guarantee diversity''.

However, though participants expressed scepticism about the residualisation process, they also appreciated the advantage score itself. ``As long as we ask the right advantage questions, we can analyse'', one stated. Another noted: ``A single score for disadvantage/need, while recognising its flaws, could provide one read of an individual's circumstances''. Furthermore, as noted above, participants requested that `advantage' be included among the information in Figure \ref{fig:impact}.

\subsection{Participant Favourites}
As part of the workshop, participants were asked to mark their favourite mock-ups. These favourites have been collated in Table \ref{tab:favourites}. We note that Mock-up 5 (Figure \ref{fig:impact}) was by-far the most favourite in both groups.

\begin{table}[htbp]
    \centering
    \caption{Participant Favourite Mock-ups}
    \label{tab:favourites}
    \begin{tabularx}{\textwidth}{lX}
        \toprule
        \textbf{Mock-up} & \textbf{Favourites} \\
        \midrule
        Mock-up 1 & 1 \\
        Mock-up 2 & 1 \\
        Mock-up 3 & 0 \\
        Mock-up 4 & 1 \\
        Mock-up 5 & 10 \\
        Mock-up 6 & 0 \\
        \bottomrule
    \end{tabularx}
\end{table}

\section{Design Recommendations and Discussion}
\subsection{Collect Participant Feedback Throughout}
One key note revealed through this process is that people would like to be involved up front in deciding what the tech should be. A strong relationship between participant feedback in interviews and their satisfaction with the mock-ups was noted.

Furthermore, both talent investment programs we worked with have created specific personas they look for. For some, these were `glue' people who helped groups function cohesively. For others, these were `diamonds in the rough', talented youth systemically undervalued due to their backgrounds. Where these aspects were included, participants showed strong interest in the mock-ups. Where they were excluded, participants often asked for these to be added.

However, it is important to note here that improving decision-making does not necessarily increase participant satisfaction with the technology. In particular, technology that makes difficult decisions less painful may be well-received, while technology that makes these decisions more salient may be less popular but more impactful.

\subsection{Be Specific About Diversity}
In the initial interviews, participants were often vague when asked to define diversity. However, when asked to expand on why diversity is important, or on what dimensions of diversity they prioritised, it became clear that `diversity' included three separate (and sometimes competing) desiderata. We have termed these: `representativeness', `different perspectives', and `boosting disadvantaged groups'. 

These three desiderata were often in tension with one another, and participants often had difficulty balancing them. However, when presented with mock-ups that specifically addressed these desiderata, participants were able to provide more specific feedback. In particular, each of these desiderata seemed to imply different program values and a different theory of change. We expound on these here:

\subsubsection{Representativeness}
Several participants seemed to believe representativeness was intrinsically valuable. This is often understood to be representativeness of society as a whole, or of the applicant pool. However, some participants also noted instrumental reasons to value representativeness. One participant noted that a team can better serve a target community if they represent that community.

Another theory from \textcite{Friedler_Scheidegger_Venkatasubramanian_2016} discusses measurement bias. Notably, if we assume that talent is equally distributed across some partition, then the most talented cohort should also be representative. However, we often observe in practice that performance is not equally distributed across these partitions. This is often due to measurement bias. In this case, representativeness is a proxy for fairness.

Finally, it might be that putting a representative group into positions of future potential ensures that when society later requires talented people, they can find one that suits their needs. Thus, representativeness is pro-social: society is better served when resources are distributed to a representative group of people.

\subsubsection{Different Perspectives}
The argument for different perspectives in the same room is often purely instrumental. While `diversity' on the whole is often spoken of as a broader benefit, the aim of placing different people in the same room is often merely to benefit the people in that room. Some participants contend that it improves cohort-level task performance, while others contend that it allows participants to better learn from each other. 

\subsubsection{Boosting Disadvantaged Groups}
The argument for boosting disadvantaged people is twofold. 

Most often, participants make a systemic critique here. That is, the world is incredibly unjust, and we want to design distribution of resources differently, but in a talent selection process we still have to operate in the unequal world. Thus, in order to correct for that injustice, we must give more resources to those who have less. This is a form of affirmative action.

However, participants also argue (perhaps relatedly) that boosting applicants from disadvantaged groups in turn allows these applicants to boost their groups. Thus, the aim of boosting disadvantaged groups is to create a more just world.

\subsection{Balance the Qualitative and the Quantitative}
Participants often noted that the mock-ups were missing key qualitative information about applicants. This qualitative information is crucial in holistic considerations of each applicant. However, when allowed to consider only qualitative information, participants obscure tradeoffs they are forced to make between different program goals. In particular, while individual-level goals are often clear, cohort-level goals (such as diversity) are easier to delay or ignore. Thus, without quantitative tools to frame the discussion, participants noted that they were often forced to make cohort-level considerations ad-hoc and towards the end of their decision-making process.

Thus, while qualitative information is crucial, it is also important to present the quantitative information necessary to make these tradeoffs salient. Ultimately, final selection decision are (and should be) made by panels of trained selectors, but in the absence of both quantitative and qualitative information to guide these decision-makers, they would be forced make decisions that are less well-informed than they could be. (In practice, both organisations we worked with make a mix of qualitatively- and quantitatively-driven decisions.)
