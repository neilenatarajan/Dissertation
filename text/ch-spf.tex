% \begin{savequote}[8cm]
% Alles Gescheite ist schon gedacht worden.\\
% Man muss nur versuchen, es noch einmal zu denken.

% All intelligent thoughts have already been thought;\\
% what is necessary is only to try to think them again.
%   \qauthor{--- Johann Wolfgang von Goethe \cite{von_goethe_wilhelm_1829}}
% \end{savequote}

\chapter{\label{ch:spf}A Possibility Frontier Approach to Diverse Talent Selection} % Maybe this is two chapters? The co-design could be one chapter, and the SPF itself could be another?

\minitoc

Organisations (e.g., talent investment programs, schools, firms) are perennially interested in selecting cohorts of talented people. And organizations are increasingly interested in selecting diverse cohorts. Except in trivial cases, measuring the tradeoff between cohort diversity and talent is computationally difficult. Thus, organizations are presently unable to make Pareto-efficient decisions about these tradeoffs. We build on disparate understandings of diversity and introduce an algorithm that approximates the upper bound on cohort talent and diversity, as measured by one of a variety of target functions capturing different desiderata. We call this object the selection possibility frontier (SPF). We then use the SPF to assess the efficiency of the selection of a talent investment program run in 2021, 2022, and 2023. We show that, in the 2021 and 2022 cycles, the program selected cohorts of finalists that could have been better along both diversity and talent dimensions (i.e., considering only these dimensions as we subsequently calculated them, they are Pareto-inferior cohorts). But, when given access to our approximation of the SPF in the 2023 cycle, the program adjusted decisions and selected a cohort on the SPF.

\section{The Diverse Talent Selection Problem}
(To-do)

\section{The Mathematics of the Selection Possibility Frontier}
(To-do)

\section{A Case Study}
(To-do)