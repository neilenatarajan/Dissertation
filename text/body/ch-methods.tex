% \begin{savequote}[8cm]
% \textlatin{Neque porro quisquam est qui dolorem ipsum quia dolor sit amet, consectetur, adipisci velit...}

% There is no one who loves pain itself, who seeks after it and wants to have it, simply because it is pain...
%   \qauthor{--- Cicero's \textit{de Finibus Bonorum et Malorum}}
% \end{savequote}

\chapter{\label{ch:methods}Methodology}

\minitoc

\section{Paradigms}
\subsection{Participatory Design}\label{ssec:participatory_design}
Participatory Design (PD) is a design paradigm in HCI emphasising active involvement of all stakeholders, particularly end-users, in the design process \cite{Hussain2014OverviewOV}. This approach recognises these users as best-positioned to speak to their subjective needs and preferences \cite{Hussain2014OverviewOV}, ensuring that designs meet user needs.

PD is a fundamentally collaborative process in which the designers and users work together to create solutions \cite{Tokranova2022ApplyingPD}. By involving users in each iteration of the development of a design, PD empowers participants to build tools that serve them \cite{Hussain2014OverviewOV}.

Historically, PD has placed special emphasis on inclusion of a broad variety of user groups \cite{Brankaert2019IntersectionsIH}. For example, \textcite{10.1145/3544549.3573821,10.1145/3544548.3580933,Chowdhury2023ReflectionsOO} employ child-centric PD. \textcite{Brankaert2019IntersectionsIH} call for (and employ) PD aiming to serve users with dementia.

This focus on inclusion, especially a focus on inclusion of oft-overlooked groups, is in itself a value-laden assumption of the PD process. That is, to employ PD is to assert the voices of the selected participants as valuable, both in the design of the tool and in the broader context of research.

\subsection{Action Research}\label{ssec:action_research}\footnote{This section reproduces text from Chapter \ref{ch:genai}.}
Action Research (AR) is a research philosophy that emphasises ``research with, rather than on, people'' \cite{bradbury_action_2003}. Rather than one specific method, AR is best seen as a collection of related methods all embodying this ethos, usually with the goal of producing research contributions useful to the target group of people \cite{lu_organizing_2023}. Among these are semiotic inspection \cite{DeSouza_Leitão_2009,Alvarado_Waern_2018} and participatory design  (PD) \cite{braun_using_2006,Griffiths_Johnson_Hartley_2007,blythe2014research,Knapp_Zeratzky_Kowitz_2016}. AR is most often used in the context of social work, but can be applied across a variety of fields \cite{dombrowski_social_2016,lu_organizing_2023}. 

In education, AR is often used in a classroom setting \cite{Mertler_2019}. \textcite{venn-wycherley_realities_2024} argue that it is crucial in this setting to perform AR on both educators (teachers) and educatees (students), as failing to do so is liable to yield contributions useful to one group but not the other. While this holds for classroom settings, engagement across the stakeholder map is less feasible or desirable in scholarship selection. Unlike teacher and student, who share the common goal that the student learn, selector and applicant are at cross purposes: practitioners seek to choose the `best' cohort of applicants (although they often disagree on what constitutes `best'), while applicants seek to be included in the chosen cohort \cite{bergman2021seven}. Thus, when elucidating the interests and desires of one group, the other will merely act as noise. (E.g., applicants who use genAI to assist in writing their application will, of course, oppose using systems that monitor genAI usage to disqualify applicants.)

AR is comparatively new to HCI \cite{Hayes_2011,lu_organizing_2023}, but its methods and philosophies closely mirror longstanding pillars of HCI \cite{Hayes_2011}. Much like PD and other HCI methods, AR seeks to democratise the research and design processes; unlike PD, AR extends beyond building solutions democratically, and sees learning through action as the ultimate research contribution \cite{Hayes_2011}. For example, AR sees all parties Become: ``Co-investigators of, co-participants in, and co-subjects of...the project'' \cite{Hayes_2011}.  Thus, research questions are formulated by and with participants, actions and interventions are designed by and with participants, and results are found by and with participants \cite{Hayes_2011}.

\subsection{Value-Sensitive Design}\label{ssec:value_sensitive_design}
Value-Sensitive Design (VSD) is a theoretically grounded design philosophy that seeks to procedurally account for human values in design \cite{batyavalue}. It, like AR, employs a three-part structure: theoretical, empirical, and technical investigations. All investigations seek to understand and incorporate human values into the design process \cite{10.1145/242485.242493}.

\subsection{The Relationship between Paradigms}
While PD and AR share the notion that research participants should be closely involved in the research process, they differ in their ontology. The two differ in the nature of their research outputs. PD is primarily concerned with the design process of products, systems, or interfaces; thus, research contributions are primarily those finalised designs \cite{zimmerman_research_2014}. AR, in contrast, is primarily concerned with the learning that occurs through action; contributions in AR are more often learnings that occurred in the act of doing \cite{Hult1980TOWARDSAD}. They also differ in the kinds of research. While PD employs an iterative design process (and often involves evaluation), each stage in this iteration is a design stage. AR, in contrast, involves cycles of three distinct activities: planning (i.e., design), acting, and reflecting \cite{Hult1980TOWARDSAD}. Finally, and most importantly, they differ in the relative role of the user. In PD, the user \emph{participates} in the act of design; a distinction remains between researchers, who facilitate discussion and implement user requirements, and participants, who discuss and require \cite{Hussain2014OverviewOV}. In AR, this distinction is elided, rendering the participant a co-researcher; it is not uncommon for participants to appear as authors in AR processes \cite{Hayes_2011}. In PD, the user is involved in the design process, while in AR, the user is involved in the research process.

VSD, unlike both PD and AR, does not primarily centre a group of users. Rather, it is a principled approach seeking to orient itself around a particular set of values \cite{10.1145/242485.242493}. Unlike both AR and PD, VSD's tripartite structure only involves the end user in one part, the empirical study \cite{10.1145/242485.242493}. This separation from the user allows the researcher to examine both the conceptual and the practical in a context removed from end-user desires.

\section{Methods} % to-do: add a table relating chapters to methods.
\subsection{Online Surveys}
The practice of running online surveys to gather quantitative data is well-established and often used both within and without HCI \cite{zhao2023fairness,pillai_adoption_2020,krishna_disagreement_2022,mai_user_nodate,bansal_does_2021,binns_its_2018,dzindolet_role_2003,papenmeier_its_2022}. Chapter \ref{ch:xai} makes use of one such survey. As is standard, we use Prolific Academic to gather participants and Formr to administer our survey \cite{binns_its_2018,Arslan_formr_2019}. More interestingly, we follow \textcite{caldwell_power_nodate} in designing our survey based on a power analysis of the statistical tests we intend to run on the output data.

\subsection{Design Workshops}
Chapters \ref{ch:xai} and \ref{ch:diversity} both make use of group design workshops to refine and evaluate design prototypes. Both follow an experience-prototype methodology \cite{Buchenau_Suri_2000}, and incorporate a few specific methodologies.

Both chapters follow \textcite{Zimmerman_Forlizzi_2017}'s scenario speed dating approach, which sees participants rapidly applying different design prototypes to (real or hypothetical) scenarios.

\textcite{Gatian_1994} has researchers asking participants to choose a favourite among a series of options as a means of comparison, while \textcite{Griffiths_Johnson_Hartley_2007} brings this method to HCI. Chapter \ref{ch:diversity} makes use of this method.

\subsection{Individual Interviews}
Chapter \ref{ch:diversity} makes use of one-on-one interviews with participants to first elucidate participant understanding of diversity. In these interviews, we incorporate a number of methods.

\textcite{Knapp_Zeratzky_Kowitz_2016}'s `crazy 8s' exercise sees participants give eight feature requests in eight minutes. Ordinarily, this exercise is done with a writing surface, but we have participants do this verbally.

\textcite{blythe2014research} introduces the concept design fiction, where participants more detail their ideal app. We adapt this to create a ``magic app'', capable of doing anything the participant desires, and asking the participant to describe this app.

\subsection{Quantitative Analysis}
Chapters \ref{ch:xai}, \ref{ch:genai}, and \ref{ch:spf} rely on a number of standard statistical tests. Primarily, we use Student's t test \cite{Mishra_Singh_Pandey_Mishra_Pandey_2019}, the Analysis of Variance (ANOVA) \cite{Mishra_Singh_Pandey_Mishra_Pandey_2019}, Pearson's test of correlation \cite{Schober_Boer_Schwarte_2018}, Tukey's Honestly Significant Difference test \cite{Kim_2015}, and the Reciever Operating Characteristic curve \cite{hanley1989receiver}. Additionally, we develop our own permutation test in Chapter \ref{ch:spf} based on \textcite{good2013permutation}.

\subsection{Qualitative Analysis}
Chapters \ref{ch:xai} and \ref{ch:diversity} engage in inductive thematic analyses of their qualitative results. In doing so, we follow the methodology introduced by \textcite{braun_using_2006} and developed in \textcite{braun_conceptual_2022,braun_toward_2023,noauthor_thematic_nodate}.

\section{Research Design}
Chapters \ref{ch:xai}, \ref{ch:genai}, \ref{ch:diversity}, and \ref{ch:spf} all detail studies conducted according to different research paradigms and employing different methodologies. Each chapter contains a self-encapsulated section on research design. However, Table \ref{tab:method_subsections} provides a high-level overview of the methods and paradigms employed in each chapter.

\begin{table}[htbp]
    \centering
    \begin{tabular}{|l|c|c|c|c|}
    \hline
    & \textbf{Chapter \ref{ch:xai}} & \textbf{Chapter \ref{ch:genai}} & \textbf{Chapter \ref{ch:diversity}} & \textbf{Chapter \ref{ch:spf}} \\
    \hline
    \textit{Participatory Design} & Yes & & Yes & \\ 
    \textit{Action Research} & & Yes & & \\ 
    \textit{Value-Sensitive Design} & & & Yes & \\ 
    \hline
    \textit{Online Surveys} & Yes & & & \\ 
    \textit{Design Workshops} & Yes & & Yes & \\ 
    \textit{Individual Interviews} & & & Yes & \\ 
    \textit{Quantitative Analysis} & Yes & Yes & & Yes \\ 
    \textit{Qualitative Analysis} & Yes & Yes & Yes & \\
    \hline
    \end{tabular}
    \caption{This table answers, for each method or paradigm and each chapter: does this methodology appear in this chapter?}
    \label{tab:method_subsections}
\end{table}
