In a globalised world, scholarship programs face political, social, internal, and moral pressures to select ``the best'' applicants. While new global scholarship initiatives aim to bear this burden, selecting fairly across regions in the modern age brings new challenges that pre-existing, low-tech decision-making systems are unequipped to handle well. To address this, we engage two global scholarship and talent investment programs in a series of studies to explore building data-driven Decision Support Tools (DSTs) for selection. Of obvious interest, both programs are interested in understanding the role generative AI (genAI) plays in essay-based evaluations and its implications for their selection decisions; we first work with selection practitioners to devise the `Decision Matrix' framework for understanding decision points on the axes of \emph{stage} (\emph{in-process} or \emph{ex-post}) and \emph{stakes} (\emph{high} or \emph{low}), place genAI-related decisions in this framework, and explore the potential for commercial detectors to serve as DSTs for these decisions. While we find that detectors lack the ability to support \emph{in-process} stage decision-making about applicants, we find useful applications \emph{ex-post}. We then apply the Decision Matrix framework to evaluate common post-hoc explainable AI (xAI) tools as DSTs. Similarly, we find post-hoc xAI limited in its application \emph{in-process}, but are able to extract valuable insights \emph{ex-post}. In attempt to design a DST for \emph{in-process} decision-making, we engage both programs in participatory design sessions to elicit their needs and preferences for a tool that could support the comparatively difficult task of incorporating diversity considerations into selection decisions. We experience-prototype six different designs and ultimately find practitioners eager to use them to support decisions across the Decision Matrix. To determine whether this eagerness reflects an actual increase in measurable outcomes (and does not just satisfy subjective practitioner desires), we implement one of the designs as a technology probe and evaluate its impact on selection outcomes; when the program uses our design, we find an increase in the selected cohort's diversity and average individual performance as measured by program metrics; this demonstrates an ability to support even high-stakes \emph{in-process} decisions. Our findings suggest a need for data-driven and AI-based DSTs in selection decisions across the Decision Matrix. We come to a notion of `Selection-Oriented AI', centred not on selection practitioners but on the aims of the selection process itself; i.e., we conclude with a call for AI- and data-driven DSTs that aim to support the aim of fair global selection by simultaneously working with selection practitioners to serve their needs and measuring program output to optimise for social benefit. % 434 words