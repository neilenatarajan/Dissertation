% \begin{savequote}[8cm]
% \textlatin{Cor animalium, fundamentum e\longs t vitæ, princeps omnium, Microco\longs mi Sol, a quo omnis vegetatio dependet, vigor omnis \& robur emanat.}

% The heart of animals is the foundation of their life, the sovereign of everything within them, the sun of their microcosm, that upon which all growth depends, from which all power proceeds.
%   \qauthor{--- William Harvey \cite{harvey_exercitatio_1628}}
% \end{savequote}

\chapter{\label{app:protocolmockup}Study Protocols}

\minitoc

\section{XAI Participatory Design Studies}\label{app:protocol}
We split our N=8 participants into two groups of 4 (G1 and G2) to run two participatory design workshops. As these are group discussions, actual programming deviates from the protocol slightly.

Our research question for both workshops is: ``Are SHAP explanations useful?''; however, to frame each workshop, we told both groups that we were interested in the answer to two questions: ``What does this technology tell us about the algorithm’s scores?'' and ``How do we envision this technology being used in future selection processes?''.

Following this, we gave both groups a 15-minute demonstration of the technology, where we described a sample case, gave some example insights, and answered any questions participants had. 

The main task for our workshops consists of hands-on cases examining a SHAP-based ``waterfall plot'' explanation of a program applicant's score. An example case can be seen in Figure \ref{fig:sample_case}. Each case is presented to participants as a single slide on a presented slideshow, with additional questions asked by the researchers to prompt discussion. We show each group 5 different cases and spend an average of 10 minutes on each case.

For each case, we ask some of the following questions to prompt discussion:

\begin{enumerate}
    \item Why are we viewing this applicant?
    \item What comments are we responding to?
    \item What does the technology appear to say about this candidate?
    \item Does the technology address the comment we are responding to?
    \item What does this case say about the algorithm as a whole?
    \item Does this case necessitate changes to the algorithm?
\end{enumerate}

Finally, after all cases had been examined, we moved to a short reflection on the technology as a whole. We asked participants to answer:

\begin{enumerate}
    \item What did you think was useful about the technology presented?
    \item What was lacking?
    \item How might this be improved?
\end{enumerate}